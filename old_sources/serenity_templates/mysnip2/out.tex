\documentclass[a4paper,9pt]{article}
\setlength{\parindent}{0pt}
\usepackage{moreverb}
\usepackage[letterpaper]{geometry}
\geometry{verbose,tmargin=1cm,bmargin=2cm,lmargin=1cm,rmargin=1cm}
\begin{document}
\begin{center}
{\Huge Serenity}
\end{center}
\tableofcontents{}


\section{Template}
\begin{verbatimtab}[4]
	#include <algorithm>
	#include <iostream>
	#include <cassert>
	#include <climits>
	#include <utility>
	#include <sstream>
	#include <iomanip>
	#include <cstdlib>
	#include <cstring>
	#include <complex>
	#include <cctype>
	#include <cstdio>
	#include <bitset>
	#include <vector>
	#include <string>
	#include <queue>
	#include <stack>
	#include <cmath>
	#include <map>
	#include <set>
	using namespace std;
	#define SET(t,v) memset((t), (v), sizeof(t))
	typedef long long LL;
	typedef long double LD;
	#define sz size()
	#define mp make_pair
	#define pb push_back
	#define x first
	#define y second
	#define Rep(i, n) for(int i = 0; i < (n); ++i)
	#define Repe(i, n) for(int i = 0; i <= (n); ++i)
	#define For(i,a,b) for(int i=(a); i<(b); i++)
	#define Fore(i,a,b) for(int i=(a); i<=(b); i++)
	#define Ford(i,a,b) for(int i=(a); i>=(b); i--)
	#define Dhelper(X) " " << #X << "='" << (X) << "'"
	#define D(X) {cerr << __LINE__ << ":" << Dhelper(X) << endl;}
	#define D2(X,Y) {cerr << __LINE__ << ":" << Dhelper(X) << Dhelper(Y) << endl;}
	#define Dv(X) {cerr << __LINE__ << ":  " << #X << " = {"; for(int __i=0;__i<(X).size();__i++) \
		cerr<<" "<<(X)[__i];cerr<<" }"<<endl;}
	#define Da(X,n) {cerr << __LINE__ << ":  " << #X << " = {"; for(int __i=0;__i<(n);__i++) \
		cerr<<" "<<(X)[__i];cerr<<" }"<<endl;}
	template<class T> inline void checkmin(T &a, const T &b) { if(a > b) a = b; }
	template<class T> inline void checkmax(T &a, const T &b) { if(a < b) a = b; }
	typedef pair<int,int> PII;
	typedef pair<LL,int> PLI;
	typedef pair<LL,LL> PLL;
	template<class T> inline T gcd(T a,T b) {if(a<0)return gcd(-a,b);if(b<0)return gcd(a,-b);
		return (b==0)?a:gcd(b,a%b);}
	template<class T> T lcm(const T &a,const T &b) { return a*(b/gcd(a,b)); }
	template<class T> inline T sqr(T X) { return X * X; }
	const long double PI=acos(-1.0L);
	#define ALL(c) (c).begin(), (c).end()
	#define RALL(c) (c).rbegin(), (c).rend()
	#define SORT(c) sort(ALL(c))
	#define RSORT(c) sort(RALL(c))
	#define UNIQUE(c) (c).resize(unique(ALL(c))-(c).begin())
	typedef vector<int> VI;
	typedef vector<LD> VLD;
	typedef vector<LL> VLL;
	typedef vector<PII> VPII;
	typedef vector<vector<int> > VVI;
	typedef vector<string> VS;
	#define Foreach(it,X) for(__typeof((X).begin())it=(X).begin();it!=(X).end();++it)
	template<class T> inline T iabs(T a) { if(a<0) return -a; return a; }
	template<class T> T Dist(T x1,T y1,T x2,T y2) { return sqrt(sqr(x1-x2)+sqr(y1-y2)); }
	template<class T> T DistSqr(T x1,T y1,T x2,T y2) { return sqr(x1-x2)+sqr(y1-y2); }
	#define iss istringstream
	#define oss ostringstream
	template<class T> string toString(T n) { oss ost; ost<<n; return ost.str(); }
	int toInt(string s){int r=0; iss sin(s); sin>>r; return r; }
	LL toLL(string s){ LL r=0; iss sin(s); sin>>r; return r; }
	double toDouble(string s){double r=0;iss sin(s);sin>>r;return r;}
	LD toLD(string s){LD r=0; iss sin(s); sin>>r; return r; }
	#define present(c,X) ((c).find(X) != (c).end())
	#define cpresent(c,X) (find(ALL(c),(X)) != (c).end())
	#define two(X) (1<<(X))
	#define contains(S,X) (((S)&two(X))!=0)
	#define twoL(X) (((long long)(1))<<(X))
	#define containsL(S,X) ((S&twoL(X))!=0)
	#define ones(X) (two(X)-1)
	#define onesL(X) (twoL(X)-1)
	template<class T> inline int countones(T n) { int c = 0; for(;n;c++) n &= (n-1); return c; }
	template<class T> inline T lowbit(T n) { return (n^(n-1))&n; }
	#define ForSubset(a,b) for(long long (a) = (b); (a)!=0; (a) = ((b) & ((a)-1)))
	template<class T> inline T getmod(T n, T m) {return (n%m+m)%m;}
	bool isleap(int year) { return (year%400==0) || ((year%4 == 0) && (year%100 != 0)); }
	template<class T> T ipow(const T &m, const T &n) { if(n==0)return 1; T a=ipow(m,n/2);
		return (n&1)?a*a*m:a*a; }
	int main() {
		#ifdef MYCOMP
		freopen("input.txt", "r", stdin);
		#endif
		return 0;
	}
\end{verbatimtab}[4]

\section{Unionfind}
\begin{verbatimtab}[4]
	struct UnionFind {
		int c;
		vector<int> parent, rank;
		UnionFind(int n) : c(n), parent(n), rank(n)
		{ for(int i = 0; i < n; ++i)	parent[i] = i, rank[i] = 0; }
		int root(int x) { if(parent[x] != x) parent[x] = root(parent[x]); return parent[x]; }
		bool join(int a, int b) {
			a = root(a); b = root(b);
			if(a == b) return false;
			if(rank[a] > rank[b]) parent[b] = a;
			else { parent[a] = b; if(rank[a] == rank[b]) rank[b]++; }
			--c;
			return true;
		 }
	};
\end{verbatimtab}[4]

\section{Getid}
\begin{verbatimtab}[4]
	map<string,int> mapid;
	int getid(string X) { return present(mapid,X)?mapid[X]:mapid[X]=mapid.size()-1; }
\end{verbatimtab}[4]

\section{SegmentTree}
\begin{verbatimtab}[4]
	const int N = 5000;
	int A[N];
	inline bool comp(int pos1, int pos2) { return A[pos1] < A[pos2]; }
	int H[4*N];
	void initialize(int n, int i, int j) {
		if(i == j) { H[n] = i; return; }
		initialize(2*n,i,(i+j)/2);
		initialize(2*n+1,(i+j)/2+1, j);
		H[n]=comp( H[2*n] , H[2*n+1] )?H[2*n]:H[2*n+1];
	}
	int query(int n, int i,int j, int a, int b) {
		if(a<=i && j<=b) return H[n];
		const int mid = (i+j)/2;
		int p1=(a>mid)?-1:query(2*n,i,mid,a,b);
		int p2=(mid>=b)?-1:query(2*n+1,mid+1,j ,a,b);
		if(p1==-1)return p2;
		if(p2==-1)return p1;
		return comp(p1,p2)?p1:p2;
	}
	void update(int n, int i, int j, int pos, int v) {
		const int mid = (i+j)/2;
		if(pos<=mid && i<mid) update(2*n,i,mid,pos,v);
		if(pos>mid && mid+1<j) update(2*n+1,mid+1,j,pos,v);
		H[n]=comp( H[2*n],H[2*n+1] )?H[2*n]:H[2*n+1];
	}
\end{verbatimtab}[4]

\section{BIT}
\begin{verbatimtab}[4]
	int bit[M],n,m;
	void update(int x, int v) {
		while( x <= n ) {
			bit[x] += v;
			x += x & -x;
		}
	}
	int sum( int x ) {
		int ret = 0;
		while( x > 0 ){
			ret += bit[x];
			x -= x & -x;
		}
		return ret;
	}
\end{verbatimtab}[4]

\section{BIT2D}
\begin{verbatimtab}[4]
	int bit[M][M];
	int n;
	int sum( int x, int y ){
		int ret = 0;
		while( x > 0 ){
			int yy = y;
				while( yy > 0 ) ret += bit[x][yy], yy -= yy & -yy;
			x -= x & -x ;
		}
		return ret ;
	}
	void update(int x , int y , int val){
		int y1;
		while (x <= n){
			y1 = y;
			while (y1 <= n){
				bit[x][y1] += val;
				y1 += (y1 & -y1);
			}
			x += (x & -x);
		}
	}
\end{verbatimtab}[4]

\section{LIS}
\begin{verbatimtab}[4]
	void LIS() {
	   int n,total, nprob = 0;
	   vector< int > table;
	   while(scanf("%d", &total)==1){
			if( total == 0 ) break;
			table.clear();
			REP(kkk, total) {
				scanf("%d",&n);
				vector< int >::iterator i = lower_bound( table.begin(), table.end(), n );
				if( i== table.end() ) table.push_back( n );
				else *i <?= n;
			}
			printf("Set %d: %d\n",++nprob,table.size());
		}
	}
\end{verbatimtab}[4]

\section{StructVec}
\begin{verbatimtab}[4]
	//vector class, also used as a point
	template<int d=2, class T=LD> struct Vec {
		T c[d];
		Vec(){ Rep(i,d) c[i]=0; }
		Vec(T input[]) { Rep(i,d) c[i]=input[i]; } //convert from array
		Vec(vector<T> input) { Rep(i,d) c[i]=input[i]; } //convert from vector
		Vec(T a1, T a2){assert(d==2); c[0]=a1;c[1]=a2; }//2d constructor
		Vec(T a1, T a2, T a3){assert(d==3); c[0]=a1;c[1]=a2;c[2]=a3; }//3d constructor
		T& operator[](int dim) { return c[dim]; } //component
		T operator[](int dim) const { return c[dim]; } //constant component
		Vec operator+(const Vec &a)const{ Vec r; Rep(i,d) r[i]=c[i]+a[i]; return r;}//addition
		Vec operator-(const Vec &a)const{ Vec r; Rep(i,d) r[i]=c[i]-a[i]; return r;}//subtraction
		T operator*(const Vec &a)const{ T r=0; Rep(i,d) r+=a[i]*c[i]; return r;}//dot product
		Vec operator*(T a)const{ Vec r; Rep(i,d) r[i]=c[i]*a; return r;}//scale vector
		T len()const { T r=0; Rep(i,d) r+=c[i]*c[i]; return sqrt(r);}//length of vector
		friend ostream& operator<<(ostream &os, const Vec &a){Rep(i,d){if(i)os<<" ";os<<a[i];}return os;}
		T operator%(const Vec<2> &a)const{assert(d==2);return c[0]*a[1]-c[1]*a[0]; }//2d cross product
		Vec operator%(const Vec<3> &a)const{assert(d==3);return Vec(c[1]*a[2]-c[2]*a[1],
			c[2]*a[0]-c[0]*a[2],c[0]*a[1]-c[1]*a[0]);}//3d cross
		Vec rotate(LD angle) {assert(d==2);return Vec(c[0]*cos(angle)-c[1]*sin(angle),
			c[0]*sin(angle)+c[1]*cos(angle));}
		bool operator<(const Vec &v)const{ Rep(i,d)if(c[i]!=v[i])return c[i]<v[i];return false; }
	};
\end{verbatimtab}[4]

\section{LinePointDist}
\begin{verbatimtab}[4]
	LD linePointDist(Vec<> A, Vec<> B, Vec<> C, bool isSegment){
		LD dist = ((B-A)%(C-A)) / sqrt((B-A)*(B-A));
		if(isSegment){
			if( (C-B)*(B-A) > eps) return (B-C).len();
			if( (C-A)*(A-B) > eps) return (A-C).len();
		}
		return iabs(dist);
	}
\end{verbatimtab}[4]

\section{SegmentsIntersect}
\begin{verbatimtab}[4]
	bool segmentsIntersect( Vec<> A, Vec<> B, Vec<> C, Vec<> E ) {
		Vec<> in = Line<>(A,B).intersect(Line<>(C,E));
		return linePointDist(A,B,in,true) < eps && linePointDist(C,E,in,true) < eps;
	}
\end{verbatimtab}[4]

\section{AreaOfPolygon}
\begin{verbatimtab}[4]
	template<class T> T twoarea(vector< Vec<2,T> > p) {
		T ret=p[p.sz-1]%p[0];
		Rep(i,p.sz-1) ret += p[i]%p[i+1];
		return ret;
	}
\end{verbatimtab}[4]

\section{StructLine}
\begin{verbatimtab}[4]
	template<class T=LD> struct Line {
		T A,B,C;
		Line(){A=B=C=0;}
		Line(T x1,T y1,T x2,T y2) { A=y2-y1; B=x1-x2; C=A*x1+B*y1; }//construct from 4 values
		T dist(T a0, T a1) const { return iabs( a0*A + a1*B - C ) / sqrt(A*A+B*B); }
		Line rot90(T x,T y)const{Line ret;ret.A=-B;ret.B=A;ret.C=ret.A*x+ret.B*y;return ret;}//rot line
		friend ostream& operator<<(ostream& os,const Line &a){return os<<a.A<<"*x + "<<a.B<<"*y = "<<a.C; }
		//requires Vec
		Line(Vec<2,T>a,Vec<2,T>b){A=b[1]-a[1];B=a[0]-b[0];C=A*a[0]+B*a[1];} //construct from two points
		T dist(Vec<2,T> a) const { return ( a[0]*A + a[1]*B - C ) / sqrt(A*A+B*B); }
		Vec<2,T> intersect(const Line &l) const {T det = A*l.B - l.A*B; if(iabs(det) < eps)det = 0;
				return Vec<2,T>((l.B*C - B*l.C)/det, (A*l.C - l.A*C)/det); }
	};
	//for segment segment intersection, check additionally
	//min(x1,x2) <= x <= max(x1,x2)
\end{verbatimtab}[4]

\section{GetMidLine}
\begin{verbatimtab}[4]
	// get a line passing between two points
	template<class T>
	Line<T> getmidline(Vec<2,T> a, Vec<2,T> b) {
		Vec<2,T> mid(a+b);
		return Line<T>(a,b).rot90(mid[0]/2,mid[1]/2);
	}
\end{verbatimtab}[4]

\section{ReflectPoint}
\begin{verbatimtab}[4]
	//reflect a point into it's "mirror" with repect to a line
	template<class T>
	Vec<2,T> reflectPoint(Vec<2,T> p, Line<T> l) {
		Line<T> r = l.rot90(p[0],p[1]);
		Vec<2,T> Y=l.intersect(r);
		return Y - (p-Y);
	}
\end{verbatimtab}[4]

\section{ConvexHull}
\begin{verbatimtab}[4]
	// Returns a list of points on the convex hull in counter-clockwise order.
	// Note: the last point in the returned list is NOT the same as the first one. Because of (k-1)
	template<class T>
	vector< Vec<2,T> > convexHull(vector<Vec<2,T> > P) {
		int n = P.size(), k = 0;
		vector< Vec<2,T> > H(2*n);
		sort(P.begin(), P.end());
		// Build lower hull
		for (int i = 0; i < n; i++) {
			while (k >= 2 && (H[k-2] - H[k-1]) % (H[k-2] - P[i]) <= 0) k--;
			H[k++] = P[i];
		}
		// Build upper hull
		for (int i = n-2, t = k+1; i >= 0; i--) {
			while (k >= t && (H[k-2] - H[k-1]) % (H[k-2] - P[i]) <= 0) k--;
			H[k++] = P[i];
		}
		H.resize( k-1 );//k-1 to remove last point (duplicate)
		return H;
	}
\end{verbatimtab}[4]

\section{Bpm}
\begin{verbatimtab}[4]
	#define M 1010
	int grid[M][M];
	int l[M], r[M], seen[M];
	int rows, cols;
	bool dfs(int x) {
		if( seen[x] ) return false;
		seen[x] = true;
		Rep(i,cols) if( grid[x][i] ) if( r[i] == -1 || dfs( r[i] ) ) {
			r[i] = x, l[x] = i;
			return true;
		}
		return false;
	}
	int bpm() {
		SET( l, -1 );
		SET( r, -1 );
		int ret = 0;
		Rep(i,rows) {
			SET( seen, 0 );
			if( dfs( i ) ) ret ++;
		}
		return ret;
	}
	bool lT[M], rT[M];
	void konigdfs(int x) {
		if( !lT[x] ) return; lT[x] = 0;
		Rep(i,cols) if(grid[x][i] && i != l[x]) {
			rT[i] = true;
			if( r[i] != -1) konigdfs(r[i]);
		}
	}
	int konig() {
		SET(lT, 1);
		SET(rT, 0);
		Rep(i,rows) if(l[i] == -1) konigdfs(i);
	}
\end{verbatimtab}[4]

\section{Hopcroft}
\begin{verbatimtab}[4]
	#define M 1000
	vector< int > edge[ M ];
	int nr, nc;
	int n, m;
	int l[M], r[M], seen[M];
	bool dfs( int x ) {
		if( seen[x] ) return false;
		seen[x] = true;
		REP(i,edge[x].sz) {
			int v = edge[x][i];
			if( r[v] == -1 || dfs( r[v] ) ) {
				r[v] = x,  l[x] = v;
				return true;
			}
		}
		return false;
	}
	int match() {
		int ret = 0;
		REP(i,n) l[i] = -1; REP(i,m) r[i] = -1;
		bool done;
		do {
			done = true;
			REP(i,n) seen[i] = 0;
			REP(i,n) if( l[i] == -1 && dfs( i ) ) done = false;
		}while( !done );
		REP(i,n) ret += ( l[i] != -1 );
		return ret;
	}
\end{verbatimtab}[4]

\section{Djikstra-set}
\begin{verbatimtab}[4]
	double d[51][51];
	double a[51][4001];
	typedef pair<double, pii> st;
	Rep(i,n) Rep(j,m) a[i][j] = 1e100;
	S.clear();
	/*add*/a[0][0] = 0.0; S.insert( make_pair(0.0, pii(0, 0) ) );
	while (!S.empty()) {
		st curr = *S.begin(), next;
		S.erase(S.begin());
		i = curr.y.x;
		k = curr.y.y;
		for (j = 0; j < n; j++) {
			next.x = curr.x + d[i][j];
			next.y.x = j;
			next.y.y = curr.y.y + get(d[i][j], maxD);
			if (next.x <= maxT && next.y.y < m && next.x < a[next.y.x][next.y.y]) {
				S.erase( st(a[next.y.x][next.y.y], next.y) );
				/*add*/a[next.y.x][next.y.y] = next.x; S.insert(next);
			}
		}
	}
\end{verbatimtab}[4]

\section{DijkstraPrqueue}
\begin{verbatimtab}[4]
	VI edge[Z], cost[Z];
	int d[Z];
	struct data {
		int u, c;
		data() {}
		data( int uu, int cc ) : u( uu ), c( cc ) {}
		bool operator < ( const data& p ) const {
			return c > p.c;
		}
	};
	int dijkstra( int na, int nb ) {
		priority_queue< data > q;
		q.push( data( na, 0 ) );
		FOR(i,1,ncity) d[i] = -1;
		d[na] = 0;
		while( !q.empty() ) {
			data u = q.top(); q.pop();
			if( d[ u.u ] < u.c ) continue;
			if( u.u == nb ) break;
			REP(i,edge[u.u].sz) {
				int v = edge[u.u][i];
				if( d[v] == -1 || d[v] > d[u.u] + cost[u.u][i] ) {
					d[v] = d[u.u] + cost[u.u][i];
					q.push( data( v, d[v] ) );
				}
			}
		}
		return d[ nb ];
	}
\end{verbatimtab}[4]

\section{ArticulationPoint}
\begin{verbatimtab}[4]
	//articulation, root is a special case
	DFS_Visit(v) { color[v]=GREY;time=time+1;d[v] = time;
		low[v]= d[v];
		for each w in Adj[v]{
			if(color[w] == WHITE){
				prev[w]=u;
				DFS_Visit(w);
				if low[w] >= d[v]
					record that vertex v is an articulation
						if (low[w] < low[v]) low[v] := low[w];
			}
			else if w is not the parent of v then
				//--- (v,w) is a BACK edge
				if (d[w] < low[v]) low[v] := d[w];
		}
		color[v] = BLACK;  time = time+1;   f[v] = time;
	}
\end{verbatimtab}[4]

\section{StronglyConnectedC}
\begin{verbatimtab}[4]
	// Inputs (populate these).
	int deg[NN]; int adj[NN][NN];
	// Union-Find.
	int uf[NN];
	int FIND( int x ) { return uf[x] == x ? x : uf[x] = FIND( uf[x] ); }
	void UNION( int x, int y ) { uf[FIND( x )] = FIND( y ); }
	// dfsn[u] is the DFS number of vertex u.
	int dfsn[NN], dfsnext;
	// mindfsn[u] is the smallest DFS number reachable from u.
	int mindfsn[NN];
	// The O(1)-membership stack containing the vertices of the current component.
	int comp[NN], ncomp;
	bool incomp[NN];
	void dfs( int n, int u ) {
	  dfsn[u] = mindfsn[u] = dfsnext++;
	  incomp[comp[ncomp++] = u] = true;
	  for( int i = 0, v; v = adj[u][i], i < deg[u]; i++ ) {
		if( !dfsn[v] ) dfs( n, v );
		if( incomp[v] ) mindfsn[u] <?= mindfsn[v];
	  }
	  if( dfsn[u] == mindfsn[u] ) {
		// u is the root of a connected component. Unify and forget it.
		do {
		  UNION( u, comp[--ncomp] );
		  incomp[comp[ncomp]] = false;
		} while( comp[ncomp] != u );
	  }
	}
	void scc( int n ) {
	  // Init union-find and DFS numbers.
	  for( int i = 0; i < n; i++ ) dfsn[uf[i] = i] = ncomp = incomp[i] = 0;
	  dfsnext = 1;
	  for( int i = 0; i < n; i++ ) if( !dfsn[i] ) dfs( n, i );
	}
\end{verbatimtab}[4]

\section{Kruskal}
\begin{verbatimtab}[4]
	sort( s, s+nedge, comp );
	unionfind u;
	int cost = 0;
	REP(i,nedge) {
		if( u.find( s[i].a ) != u.find( s[i].b ) ) {
			cost += s[i].c;
			u.Union( s[i].a , s[i].b );
		}
	}
\end{verbatimtab}[4]

\section{Maxflow}
\begin{verbatimtab}[4]
	#define MM 1211
	VI edges[MM];
	int visited[MM];
	int cap[MM][MM];
	int dfs(int src, int end, int fl) {
		if(visited[src])return 0;
		if(src==end)return fl;
		visited[src]=1;
		Foreach(v,edges[src]) {
			int x = min( fl, cap[src][*v] );
			if(x>0) {
				x = dfs(*v,end,x);
				if(x==0)continue;
				cap[src][*v] -= x;
				cap[*v][src] += x;
				return x;
			}
		}
		return 0;
	}
	int flow(int src, int sink) {
		int ret = 0;
		while(1) {
			SET(visited,0);
			int d = dfs(src,sink,INT_MAX);
			if(d==0)break;
			ret += d;
		}
		return ret;
	}
	void addEdge(int from, int to, int value) {
		cap[ from ][to]=value;
		edges[from].pb(to);
		edges[to].pb(from);
	}
\end{verbatimtab}[4]

\section{MaxflowDinic}
\begin{verbatimtab}[4]
	#include<limits>
	#define NN 5010
	const long long INF = numeric_limits<long long>::max();
	struct edge { int point, next; LL flow, capa; };
	vector<edge> edges;
	int head[NN], dist[NN], Q[NN], work[NN];
	int node, src, dest;
	void flowinit(int _node,int _src,int _dest) {
		node=_node;
		src=_src;
		dest=_dest;
		SET(head,-1);
		edges.clear();
		edges.reserve( 60010 );
	}
	bool dinic_bfs() {
		SET(dist,-1); dist[src]=0;
		int sizeQ=0;
		Q[sizeQ++]=src;
		for (int cl=0;cl<sizeQ;cl++)
			for (int k=Q[cl],i=head[k];i>=0;i=edges[i].next)
				if (edges[i].flow<edges[i].capa && dist[edges[i].point]<0)
					dist[edges[i].point]=dist[k]+1, Q[sizeQ++]=edges[i].point;
		return dist[dest]>=0;
	}
	LL dinic_dfs(int x, LL exp) {
		if (x==dest) return exp;
		for (int &i=work[x];i>=0;i=edges[i].next) {
			int v=edges[i].point;
			LL tmp;
			if (edges[i].flow<edges[i].capa && dist[v]==dist[x]+1
					&& (tmp=dinic_dfs(v,min(exp,edges[i].capa-edges[i].flow)))>0)
				return edges[i].flow+=tmp,edges[i^1].flow-=tmp, tmp;
		}
		return 0;
	}
	LL dinic_flow() {
		LL result=0;
		while (dinic_bfs()) {
			Rep(i,node) work[i]=head[i];
			for(LL delta; delta=dinic_dfs(src,INF); result+=delta);
		}
		return result;
	}
	void addEdge(int u,int v, LL c1=1, LL c2=0) {
	//	cout << "add edge " << u+1 << " to " << v+1 << endl;
		edges.pb( (edge) { v,head[u],0,c1 } ); head[u] = edges.sz-1;
		edges.pb( (edge) { u,head[v],0,c2 } ); head[v] = edges.sz-1;
	}
\end{verbatimtab}[4]

\section{MincostMaxflow}
\begin{verbatimtab}[4]
	#define N 705
	int n, nE;
	int d[N], pre[N];
	struct edge {
		int to, cost, cap;
		int back;
	};
	edge E[N*N];
	vector< int > e[N];
	int mincost( int s, int t, int lim ) {
		int flow = 0, ret = 0;
		while( flow < lim ) {
			SET( d, -1 ); SET( pre, -1 );
			d[s] = 0;
			// bellman ford
			int jump = n-1;
			bool done = 0;
			while( !done && --jump >= 0) {
				done = 1;
				REP(i,n) if( d[i] != -1 ) REP(j,e[i].sz) {
					edge& x = E[ e[i][j] ];
					int v = x.to;
					if( x.cap > 0 && ( d[v] == -1 || d[v] > d[i] + x.cost )) {
						d[v] = d[i] + x.cost;
						pre[v] = x.back;
						done = 0;
						//cout<<v<<" "<<d[v]<<endl;
					}
				}
				if( done ) break;
			}
			if( d[t] == -1 ) break;
			// traverse back
			int x = t, cflow = 1<<30;
			while( x != s ) {
				edge& ed = E[ pre[x] ];
				cflow = min( cflow, E[ ed.back ].cap );
				x = ed.to;
			}
			if( !cflow ) break;
			int take = min( lim - flow, cflow );
			ret += d[t] * take;
			flow += take;
			x = t;
			while( x != s ) {
				edge& back = E[ pre[x] ];
				edge& forw = E[ back.back ];
				back.cap += take;
				forw.cap -= take;
				x = back.to;
			}
		}
		if( flow < lim ) return -1;
		return ret;
	}
	// remember to add -cost in the opposite direction
	void add( int u, int v, int uv, int vu, int fuv, int fvu ) {
		int a = nE, b = nE+1;
		nE += 2;
		E[ a ].to = v, E[ a ].cost = uv, E[ a ].cap = fuv, E[ a ].back = b;
		E[ b ].to = u, E[ b ].cost = vu, E[ b ].cap = fvu, E[ b ].back = a;
		e[ u ].pb( a ), e[ v ].pb( b );
	}
\end{verbatimtab}[4]

\section{BigNumInt}
\begin{verbatimtab}[4]
	const int B = 10;// If you change this value, then you
	                 // will also have to change operator<<.
	class num : public vector<char>
	{public:
		num(){}
		num(int n) {
	        for(;n;n/=B) push_back(n%B);
	    }
		num(string n) {
			int down=0,sign=1;
			if(n[down]=='-') {
				down++;
				sign=-1;
			}
			for(int i=n.size()-1;i>=0;i--)
				push_back(sign*(n[i]-'0'));
			norm();
		}
		void norm() {
			while(!empty()&&!back())
	            pop_back();
			int r = 0;
			for(int i=0;i<size();i++) {
				at(i) += r;
				r = at(i) / B;
				at(i) %= B;
			}
			for(;r;r/=B) push_back(r);
		}
	};
	ostream& operator<<(ostream &s, const num &a) {
		if(!a.size()) return s << '0';
	    if(a.back()<0)s << '-';
		for(int i = a.size()-1;i>=0;i--)
			s << char(iabs(a[i])+'0');
		return s;
	}
	bool operator==(const num &a, const num &b) {
		if(a.size()!=b.size())return false;
		for(int i=0;i<a.size();i++)
			if(a[i]!=b[i])return false;
		return true;
	}
	bool operator!=(const num &a, const num &b) {
	    return !(a==b);
	}
	bool operator<(const num &a, const num &b) {
	    if(!a.size() && !b.size())return false;
	    if(!a.size()) return 0 < b.back();
	    if(!b.size()) return a.back() < 0;
	    if(a.back()*b.back()<0)return a.back()<0;
		if(a.size()!=b.size())return (a.size()<b.size())^(a.back()<0);
		for(int i=a.size()-1;i>=0;i--)
			if(a[i]!=b[i])return a[i]<b[i];
		return false;
	}
	bool operator<=(const num &a, const num &b) {
	/*    if(!a.size() && !b.size())return true;
	    if(!a.size()) return 0 <= b.back();
	    if(!b.size()) return a.back() <= 0;
	    if(a.back()*b.back()<0)return a.back()<0;
		if(a.size()!=b.size())return (a.size()<b.size())^(a.back()<0);
		for(int i=a.size()-1;i>=0;i--)
			if(a[i]!=b[i])return a[i]<=b[i];
		return true;*/
		return a < b || a == b; //inefficient?
	}
	num operator-(const num &a) {
	    num b(a);
	    for(int i=0;i<b.size();i++) b[i]=-b[i];
	    return b;
	}
	num operator-(const num &a, const num &b);
	num operator+(const num &a, const num &b) {
	    if(!a.size())return b;
	    if(!b.size())return a;
	    if(a.back()*b.back()<0)return a -(-b);
		num c(a);
		int s = min(a.size(),b.size());
		for(int i=0;i<s;i++)
	        c[i]+=b[i];
		while(s<b.size())
	        c.push_back(b[s++]);
		c.norm();
		return c;
	}
	num operator-(const num &a, const num &b) {
	    if(!a.size())return -b;
	    if(!b.size())return a;
	    if(a.back()*b.back()<0)return a +(-b);
	    num up=iabs(a),down=iabs(b);
	    if(up<down)swap(up,down);
	    for(int i=0;i<down.size();i++)
	        up[i]-=down[i];
	    for(int i=0;i<up.size();i++)
	        for(;up[i]<0;up[i]+=B)
	            up[i+1]--;
		up.norm();
		return (a<b)?-up:up;
	}
	num operator*(const num &a, const num &b) {
	    num c;
	    c.resize(a.size()+b.size());
	    fill(c.begin(),c.end(),0);
	    for(int i=0;i<a.size();i++)
	        for(int j=0;j<b.size();j++) {
	            int r = c[i+j] + int(a[i])*b[j];
				c[i+j+1] += r/B;
				c[i+j] = r%B;
			}
		c.norm();
	    return c;
	}
	num divByInt(const num &a, int b) {
		num c(a);
	    int r=0;
		for(int i=c.size()-1;i>=0;i--) {
	        c[i] += r*B;
	        r = c[i] % b;
	        c[i] /= b;
	    }
		c.norm();
		return c;
	}
	num operator/(const num &a, const num &b) {
	    if(a==0 || b==0) return 0;
		num c;
		c.resize(a.size());
	    num row,bAbs=iabs(b),aAbs=iabs(a);
		for(int i=a.size()-1;i>=0;i--) {
	        row.insert(row.begin(),1,aAbs[i]);
			row.norm();
	        for(c[i]=0;bAbs <= row;c[i]++)
	            row = row - bAbs;
	    }
		c.norm();
		return (a.back()*b.back()<0)?-c:c;
	}
	num operator%(const num &a, const num &b) {
	    if(a==0 || b==0) return 0;
	    num row,bAbs=iabs(b),aAbs=iabs(a);
		for(int i=a.size()-1;i>=0;i--) {
	        row.insert(row.begin(),1,aAbs[i]);
			row.norm();
	        while(bAbs <= row)
	            row = row - bAbs;
	    }
	    return (a.back()<0)?-row:row;
		//return a - (a/b)*b;//unefficient?
	}
	num power(const num &m, const num &n) {
		if(n.empty())return 1;
		num a=power(m,n/2);
		return (n[0]%2)?a*a*m:a*a;
	}
\end{verbatimtab}[4]

\section{BsearchLE}
\begin{verbatimtab}[4]
	int down = start;
	int up = end+1;
	while(down+1 < up) {
	    int mid = (down+up)/2;
	    if(f1(mid)) down = mid; else up = mid;
	}
	assert(down == L);
\end{verbatimtab}[4]

\section{BsearchGE}
\begin{verbatimtab}[4]
	int down = start-1;
	int up = end;
	while(down+1 < up) {
	    int mid = (down+up)/2;
	    if(f2(mid)) up = mid; else down = mid;
	}
	assert(down+1 == L);
\end{verbatimtab}[4]

\section{PermIndex}
\begin{verbatimtab}[4]
	int perm_index (char pit[], int size) {
		int i;
		register int j, ball;
		int index = 0;
		for (i = 1; i < size; i++) {
			ball = pit[i-1];
			for (j = i; j < size; j++) {
				if (ball > pit[j])
				index ++;
			}
			index *= size - i;
		}
		return index;
	}
\end{verbatimtab}[4]

\section{Geodistance}
\begin{verbatimtab}[4]
	double gdist
	( double a_lat, double b_lat, double a_long, double b_long ) {
		return acos(cos(a_lat) * cos(b_lat) * cos(a_long - b_long) + sin(a_lat) * sin(b_lat));
	} // don't forget to multiply radius with it! ;)
\end{verbatimtab}[4]

\section{Matrixmul}
\begin{verbatimtab}[4]
	i64 mod;
	struct matrix {
		int m,n;
		i64 a[10][10];
	};
	void print( matrix& a ) {
		REP(i,a.m) {
			REP(j,a.n) cout <<" " << a.a[i][j];
			cout << endl;
		}
	}
	matrix operator * (  matrix& x,  matrix& y ) {
		matrix ret;
		ret.m = x.m;
		ret.n = y.n;
		REP(i,ret.m) REP(j,ret.n) {
			ret.a[i][j] = 0;
			REP(k, x.n) {
				ret.a[i][j] += (x.a[i][k] * y.a[k][j]);
				ret.a[i][j] %= mod;
			}
		}
		return ret;
	}
	matrix power( matrix& x , int n ) {
		if( n == 1 ) return x;
		matrix ret = power( x, n/2 );
		ret = ret * ret;
		if( n % 2 == 0 ) return ret;
		ret = ret * x ;
		return ret;
	}
\end{verbatimtab}[4]

\section{Heron}
\begin{verbatimtab}[4]
	double area( double x, double y, double z ) {
		double s = ( x + y + z ) / 2.;
		double X = s * ( s - x ) * ( s - y ) * ( s - z );
		return sqrt( X );
	}
\end{verbatimtab}[4]

\section{Genprimes}
\begin{verbatimtab}[4]
	#define MAXP 1000000
	bool primes[MAXP];
	vector<int> plist;
	void genprimes() {
		plist.clear();
		int m = (int)sqrt(int(MAXP));
		primes[0]=primes[1]=0;
		for(int i=2;i<MAXP;i++) primes[i] = 1;
		for(int i=2;i<=m;i++)
			if (primes[i])
				for(int j=i*i;j<MAXP;j+=i) primes[j] = 0;
		for(int i=2;i<MAXP;i++)if(primes[i])plist.push_back(i);
	}
\end{verbatimtab}[4]

\section{Choose}
\begin{verbatimtab}[4]
	LL choose[50][50];
	For(i,1,50) {
		choose[i][0]=choose[i][i]=1;
		For(j,1,i) {
			choose[i][j] = choose[i-1][j]+choose[i-1][j-1];
		}
	}
\end{verbatimtab}[4]

\section{EuclidExtended}
\begin{verbatimtab}[4]
	template<class T> inline T euclid(T a,T b,T &X,T &Y) {
		if(a<0)	{ T d=euclid(-a,b,X,Y); X=-X; return d;	}
		if(b<0) { T d=euclid(a,-b,X,Y); Y=-Y; return d; }
		if(b==0) { X=1; Y=0; return a; }
		else{
			T d=euclid(b,a%b,X,Y); T t=X;
			X=Y; Y=t-(a/b)*Y;
			return d;
		}
	}
\end{verbatimtab}[4]

\section{Isprime2MillerRabin}
\begin{verbatimtab}[4]
	bool isprime2(long long p, int iteration){
	    if(p<2)
	        return false;
	    if(p!=2 && p%2==0)
	        return false;
	    long long s=p-1;
	    while(s%2==0)
	        s/=2;
	    for(int i=0;i<iteration;i++){
	        long long a=rand()%(p-1)+1,temp=s;
	        long long mod=powermod(a,temp,p);
	        while(temp!=p-1 && mod!=1 && mod!=p-1){
	            mod=multiplymod(mod,mod,p);
	            temp *= 2;
	        }
	        if(mod!=p-1 && temp%2==0)
	            return false;
	    }
	    return true;
	}
\end{verbatimtab}[4]

\section{EulerPhi}
\begin{verbatimtab}[4]
	template<class T> T euler(T n) {
		T result = n;
		for(int i=2;i*i <= n;i++) {
			if (n % i == 0) result -= result / i;
			while (n % i == 0) n /= i;
		}
		if (n > 1) result -= result / n;
		return result;
	}
\end{verbatimtab}[4]

\section{Euler2Phi}
\begin{verbatimtab}[4]
	template<class T> inline T euler2(T n) {
		vector<pair<T,int> > R=factorize(n);
		T r=n;
		for (int i=0;i<R.size();i++)
			r=r/R[i].first*(R[i].first-1);
		return r;
	}
\end{verbatimtab}[4]

\section{SGcdCongrModinv}
\begin{verbatimtab}[4]
	LL gcd( LL a, LL b ) {
		return !b ? a : gcd( b, a%b );
	}
	PII egcd( LL a, LL b ) {  // returns x,y | ax + by = gcd(a,b)
		if( b == 0 ) return mp( 1, 0 );
		else {
			PII d = egcd( b, a % b );
			return mp( d.y, d.x - d.y * ( a / b ) );
		}
	}
	LL congruence( int a, int b, int n ) { // finds ax = b(mod n)
		int d = gcd( a, n );
		if( b % d != 0 ) return 1<<30; // no solution
		PII ans = egcd( a, n );
		LL ret = ans.x * ( b/d + 0LL ), mul = n/d;
		ret %= mul;
		if( ret < 0 ) ret += mul;
		return ret;
	}
	LL m_inverse( LL num, LL mod ) {
		PII p = egcd( num, mod );
		int ret = p.x % mod;
		if( ret < 0 ) ret += mod;
		return ret;
	}
\end{verbatimtab}[4]

\section{Sterling1}
\begin{verbatimtab}[4]
	LL S[110][110]; LL fact[110];
	fact[0] = 1; For(i,1,110) fact[i] = (fact[i-1] * i) % M;
	SET(S,0);
	For(n,1,110) {
		S[n][n] = 1; S[n][1] = fact[n-1];
		if(n%2 == 0) S[n][1] *= -1;
		For(k,2,n) {
			S[n][k] = ( - (n-1)*S[n-1][k] + S[n-1][k-1] )%M;
		}
	}
	if(K==0)
		cout << (N==0) << endl;
	else
		cout << iabs(S[N][K]%M) << endl;
\end{verbatimtab}[4]

\section{Sterling2}
\begin{verbatimtab}[4]
	LL S[110][110]; LL fact[110];
	fact[0] = 1; For(i,1,110) fact[i] = (fact[i-1] * i) % M;
	SET(S,0);
	For(n,1,110) {
		S[n][n] = 1; S[n][1] = 1;
		For(k,2,n) {
			S[n][k] = ( (k)*S[n-1][k] + S[n-1][k-1] )%M;
		}
	}
	if(K==0)
		cout << (N==0) << endl;
	else
		cout << iabs(S[N][K]%M) << endl;
\end{verbatimtab}[4]

\section{Totient}
\begin{verbatimtab}[4]
	FOR(i,1,M) f[i] = i;
	FOR(n,2,M) if( f[n] == n ) for(int k=n; k<=M; k+=n) f[k] *= n-1, f[k] /= n;
\end{verbatimtab}[4]

\section{SieveBits}
\begin{verbatimtab}[4]
	const int MAX = 100000000;
	int p[ MAX/64 + 2 ];
	int np = 0;
	#define on(x) ( p[x/64] & (1<<( (x%64)/2 ) ) )
	#define turn(x)  p[x/64] |= (1<<( (x%64)/2 ) )
	void sieve() {
		for(int i=3;i*i<MAX; i+=2) {
			if( !on(i) ) {
				int ii = i*i;
				int i2 = i+i;
				for(int j=ii; j<MAX; j+=i2) turn(j);
			}
		}
	}
	inline bool prime( int num ) {
		return num > 1 && ( num == 2 || ( (num&1) && !on( num ) ) );
	}
\end{verbatimtab}[4]

\section{KmpExplained}
\begin{verbatimtab}[4]
	f[ len ] = the longest suffix that we can still use for matching if its mismatched at len
	f[0] = f[1] = 0;
	FOR(i,2,len) {
		int j = f[i-1]; // think as a recursion - last best matching
		while( true ) {
			if( s[j] == s[i-1] ) { // we got a way to maximize
				f[i] = j + 1;
				break;
			}else if( !j ) { // suicide
				f[i] = 0;
				break;
			}else j = f[j]; // think recursively, we couldn't use s[i-1] to improve the match for j
			//now we'll go for the best suffix of j and check if we can improve that by using s[i-1]
		}
	}
	i = j = 0;
	while( true ) {
		if( i == len ) break;
		if( text[i] == s[j] ) { // we got an improve
			i++, j++;
			if( j == slen ) // match found
		}else if( j > 0 ) j = f[j]; // then we'll check again if we can improve at i
		else i++;
	}
\end{verbatimtab}[4]

\end{document}
